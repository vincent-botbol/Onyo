\documentclass[11pt,a4paper]{report}

%\parskip = \baselineskip

\usepackage[utf8]{inputenc}
\usepackage[frenchb]{babel}
\usepackage[T1]{fontenc}

\usepackage[hidelinks]{hyperref}
\usepackage{tikz}
\usetikzlibrary{arrows,positioning,trees} 
\tikzset{
    %Define standard arrow tip
    >=stealth',
    %Define style for boxes
    punkt/.style={
           rectangle,
           rounded corners,
           draw=black, very thick,
           text width=6.5em,
           minimum height=2em,
           text centered
    },
    punkt2/.style={
      rectangle,
      draw=black, very thick,
      text width=6.5em,
      minimum height=2em,
      text centered
    },
    % Define arrow style
    pil/.style={
           ->,
           thick,
           shorten <=2pt,
           shorten >=2pt,}
}

\usepackage{listings}
\usepackage{color}
\definecolor{lightgray}{rgb}{.9,.9,.9}
\definecolor{darkgray}{rgb}{.4,.4,.4}
\definecolor{purple}{rgb}{0.65, 0.12, 0.82}
\lstdefinelanguage{JavaScript}{
  keywords={typeof, new, true, false, catch, function, return, null, catch, switch, var, if, in, while, do, else, case, break},
  keywordstyle=\color{blue}\bfseries,
  ndkeywords={class, export, boolean, throw, implements, import, this},
  ndkeywordstyle=\color{darkgray}\bfseries,
  identifierstyle=\color{black},
  sensitive=false,
  comment=[l]{//},
  morecomment=[s]{/*}{*/},
  commentstyle=\color{purple}\ttfamily,
  stringstyle=\color{red}\ttfamily,
  morestring=[b]',
  morestring=[b]"
}
\lstnewenvironment{JavaScript}
                  {\lstset{%
                      language=JavaScript,
                      %backgroundcolor=\color{lightgray},
                      extendedchars=true,
                      basicstyle=\footnotesize\ttfamily,
                      showstringspaces=false,
                      showspaces=false,
                      numbers=left,
                      numberstyle=\footnotesize,
                      numbersep=9pt,
                      tabsize=2,
                      breaklines=true,
                      showtabs=false,
                      captionpos=b%
                    }
                  }
                  {}

\lstnewenvironment{OCaml}
                  {\lstset{
                      language=[Objective]Caml,
                      breaklines=true,
                      commentstyle=\color{purple},
                      stringstyle=\color{red},
                      identifierstyle=\ttfamily,
                      keywordstyle=\color{blue},
                      basicstyle=\footnotesize
                    }  
                  }
                  {}

\title{Rapport de stage\\Programmation Enyo en OCaml}
\author{Vincent Botbol}
\date\today

\begin{document}
\maketitle

\chapter{Introduction}

En quelques années, les appareils mobiles se sont peu à peu imposés sur le marché technologique.
Grâce à leurs interfaces graphiques simplifiées et leurs fonctionnalités, ces outils ont sû se 
rendre indispensable auprès du grand-public.
Aujourd'hui, environ un tiers de la population possède un smartphone ou une tablette.

La variété de ces appareils est telle que la tâche est parfois agaçante pour les développpeurs 
d'exporter leurs applications sur d'autres systèmes utilisant des langages différents.
La nécessité d'une unification s'est donc imposé. Pour se faire, le choix des constructeurs quant à
la technique s'est dirigé vers les nouvelles technologies web :
HTML5 et JavaScript. Populaire et en constante évolution, ces deux solutions apportent un grand nombre
de possibilités mais possèdent néanmoins leurs désavantages.

Le but de ce stage est de permettre au programmeur de développer des applications portables de manière sûre
en passant pour cela par le langage OCaml garant d'une solidité à travers la puissance de son typage statique
tout en gardant le confort de programmation qu'on lui connait.
Pour se faire, on s'est intéressé à la bibliothèque JavaScript ``Enyo'' (\ref{chap:enyo}) 
en interfaçant celle-ci afin de satisfaire notre objectif.

La traduction d'OCaml vers JavaScript a déjà été mise-en-oeuvre dans plusieurs travaux : O'Browser
et Js\_of\_ocaml. Fournissant des bases intéressantes pour ce projet, il a été pratique d'en faire usage
pour s'abstraire du problème d'inter-opérabilité entre les deux langages.

La divergence entre JavaScript et OCaml est importante. Cela peut beaucoup jouer lors d'un tel 
interfaçage. Il est alors important de bien étudier la bibliothèque afin de déterminer le niveau
d'utilisation des traits intrusifs.
En répondant cette question, il faut pouvoir déterminer le type de modèle et sa représentation
s'adaptant le mieux.
Les objets Enyo étant nombreux, il peut être important de se donner les moyens de les représenter 
de façon efficace. Se munir d'un IDL peut-il être un atout pour réaliser ceci?

Premièrement, nous étudierons Enyo pour nous permettre d'obtenir les informations nécessaires 
pour établir une approche.

Ensuite, nous tenterons de fournir un modèle en détaillant les raisons de ce choix et les problèmes
qui en ressortent.

Enfin, nous présenterons un IDL permettant de modéliser les objets Enyo.

%%%%%%%%%%%%%%%%%%
%   CHAPITRE 1   %
%%%%%%%%%%%%%%%%%%
\chapter{Présentation Enyo}\label{chap:enyo}
\section{Introduction}
Enyo\cite{enyo} est un framework Javascript orienté-objet. Il privilégie un style de développement basé
sur une encapsulation arborescente de composants.\medskip

Dans sa première version, Enyo fut employé pour le développement d'application des 
tablettes HP TouchPad fonctionnant sous webOS. Après un succès mitigé,
HP décida de transformer webOS en projet \emph{open source} devenant ainsi : Open WebOS.\medskip

Enyo passe alors en version 2.0. Sa conception en Javascript profite de  
l'engouement actuel pour ce langage et permet aux développeurs de déployer leurs 
applications sur les plateformes le supportant, ce qui aujourd'hui désigne 
la plupart des appareils grand public (Android, iOS, WindowsPhone, etc.)\medskip

On peut attribuer à Enyo des avantages tels que sa prise en main aisée grâce au style 
de programmation proche d'une vision du DOM\footnote{Document Object Model}, le poids relativement 
faible de sa bibliothèque et sa possibilité d'extension relativement aisée.

\clearpage

\section{Débuter en Enyo}

Il est important de distinguer trois phases dans le développement d'une application 
Enyo.
\begin{itemize}
\item La première, dite de déclaration, consiste à définir les composants, leurs positions dans 
la hiérarchie de l'application ainsi que le modèle de calcul.
\item La seconde est celle de la génération. Permettant à la bibliothèque Enyo de 
``compléter'' le squelette décrit par la phase de déclaration, l'objet destiné
à être instancié.
\item Enfin, nous avons la phase d'instanciation où le résultat de la génération
est à même d'être créé puis utlisable, par exemple, en l'affichant.
\end{itemize}\medskip

On observe ainsi de fortes similitudes avec un langage de description d'interface.
Il faut commencer par \emph{décrire} l'objet, ensuite le \emph{générer} 
pour enfin l'\emph{utiliser}.

La définition des composants de l'application consiste à déclarer 
des objets Javascript possédant des propriétés qui auront un sens au moment d'étendre
l'application.\smallskip

Par exemple, définissons un simple morceau de texte contenant la chaine ``Bonjour monde!'':
\begin{JavaScript}
  var monApplication = {kind: "Control",
                        name: "MonApp",
                        content: "Bonjour monde!"}
\end{JavaScript}

Ici, nous avons simplement déclaré un objet Javascript\footnote{%
Pour simplifier les déclarations d'objets, nous utilisons la 
notation Json de JavaScript.}
muni de propriétés suffisantes pour renseigner Enyo sur le type d'objet à créer.\\
Explicitons ces propriétés :
\begin{itemize}
\item \emph{kind} définit le type d'objet Enyo que nous souhaitons créer.\smallskip
\item \emph{name} nomme l'objet que nous sommes en train de créer. C'est par la
valeur de cette propriété que nous pourrons accéder à un composant ou encore l'instancier.
\footnote{Il est important de noter que si placée dans un composant en racine de l'application,
la propriété \emph{name} déterminera le nom de la classe générée. Autrement, le nom d'un sous-composant
servira de clé à une table d'association définie sur le composant racine de l'application 
servant à récupérer les divers sous-composants
}\smallskip
\item \emph{content} attribut un texte au composant. Cette propriété concerne les objets Enyo de type 
``Control'' ou un de ses descendants héritant de celle-ci.\smallskip
\end{itemize}\medskip

Désormais défini, il faut passer le composant à la méthode \emph{kind} d'Enyo pour générer la nouvelle
classe de l'application.

\begin{JavaScript}
  enyo.kind(monApplication);
\end{JavaScript}

A la suite de cet appel, on pourra constater qu'une nouvelle classe \mbox{\emph{MonApp}} 
a été ajouté à l'environnement global de JavaScript. Il ne reste donc plus qu'à l'instancier 
pour pouvoir ensuite l'afficher.

\begin{JavaScript}
  var app = new MonApp();
  app.renderInto(document.body);
\end{JavaScript}

En s'intéressant de plus près à l'objet ``app'', on peut constater qu'il
possède un nombre important de méthodes et de propriétés que nous n'avons 
pas définies à la description, telle que la méthode \emph{renderInto}
que nous venons d'utiliser.

Ces méthodes ont été ajoutées par induction lors de la génération de 
l'application par la méthode \emph{enyo.kind}.

Si le champ \emph{kind} spécifie un ``Control'', il est donc nécessaire à Enyo de
lui rajouter toutes les propriétés et méthodes que possède un Control Enyo qui
n'ont pas été définies par l'utilisateur.

\section{Ajout de sous-composants}
Enyo suivant une logique d'arborescence de composants. Lorsque l'on crée 
son application, il est nécessaire de pouvoir lui ajouter des sous-composants. 
On réalise cela en définissant la propriété \emph{components} avec en valeur le tableau 
des composants que nous souhaitons.

Imaginons ici que l'on souhaite créer une application consistant d'un champ de saisie 
et d'un bouton de validation:
\begin{JavaScript}
 enyo.kind({kind:"Control",
            name:"MonApp",
            components: [
                         {kind:"Input"},
                         {kind:"Button", content:"ok"}
                        ]
           });
\end{JavaScript}

La fonction kind va récursivement créer tous les sous-composants de la hiérarchie 
ainsi que leur méthodes associées.

Comme précédemment, pour l'afficher, on instancie la nouvelle classe générée 
et on lui applique une méthode d'affichage : 

\begin{JavaScript}
  var app = new MonCadre();
  app.renderInto(document.body);
\end{JavaScript}

Si l'on observe le code html après ces appels, on s'aperçoit qu'enyo a injecté dans 
le DOM une balise html de type <div> dans laquelle sera placée, respectivement, 
deux balises : un <input type="text" ..> et un <button>.

Notons que nous n'avons pas défini la propriété \emph{name} des sous-composants. 
Rappelons que le nom d'un composant sert uniquement à l'accession de celui-ci par les 
autres composants de la hiérarchie ou encore pour son instanciation. Cependant, lors de
l'appel à \emph{enyo.kind}, seule la racine de l'arbre sera étendu à l'environnement.

Il faut aussi noter que si la propriété name n'est pas définie, Enyo s'en charge en lui donnant 
un label géneré (``input0'' et ``button0'' ici). Il n'est évidemment pas recommandé 
d'utiliser ces derniers pour y accéder.

\section{\'Evénements}\label{sec:event}

Reprenons l'exemple précédent et supposons désormais que l'on souhaite, lorsque l'on clique sur le bouton, afficher un message d'alerte.

\begin{JavaScript}
enyo.kind({kind:"Control",
           name:"MonCadre",
           components: [
             {kind:"Input", value:""},
             {kind:"Button", content:"ok",
              handlers:{ontap:"monCallback"},
              monCallback:function(sender, event){
                alert("ok");
                return true;
              }
             }
           ]
       	  });
\end{JavaScript}

Au bouton, nous avons rajouté une propriété \emph{handlers}. 
Cet objet sert à définir tous les événements que l'objet va traiter. 
Cela peut-être des événements du DOM (onkeyup, onload, ...) comme des événements 
spécifiques à la plateforme sur laquelle l'application est déployée 
(l'inclinaison d'une tablette par exemple). Ici, nous utilisons l'événement "ontap", alias d'"onclick".\\

A chaque propriété-événement est attachée une chaine contenant le nom de la fonction à appeler.

Nous définissons donc la fonction ``click'' qui prend en premier argument l'émetteur de 
l'événement et l'événement envoyé contenant ses propriétés.

La valeur de retour de cette fonction spécifie si l'événement doit s'arrêter ou 
remonter (``bubble'') aux composants parents. On retourne vrai pour l'empêcher de 
remonter et faux pour le laisser ``buller'' sur ses ancêtres.

On notera qu'en général, les fonctions de traitement ne pas censées devoir remonter l'arbre
sont définies à la racine qui possède un accès à tous ses sous-composants 
grâce à sa table de hash ``\$''.

Il existe quelques spécificités pour la manipulation d'événements :
\begin{itemize}
\item Si la fonction de traitement n'est pas trouvée, l'événement est remonté dans l'arbre de 
  composants jusqu'à trouver une fonction dont le nom correspond.
\item Il n'est nécessaire de déclarer le bloc \emph{handlers} que lorsque la fonction de traitement
  se situe dans le même objet que celui-ci. Par exemple, cette définition est parfaitement valable :
  \begin{JavaScript}
    enyo.kind({kind:"Control", 
               components:[{kind:"Bouton", 
                            ontap:"traiteClique"}
                          ],
               traiteClique:function(sender, event){
                 alert("Click recu"); return true
               }
    });
  \end{JavaScript}
\item Il est possible de propager un événement manuellement en utilisant la fonction 
  \emph{send(nomEvenement, evenement)} de l'objet \emph{enyo.Signals}. 
  Cet événement est remis à tous les objets de type \emph{kind:"Signals"}. Il suffit alors
  à l'objet-récepteur de posséder un sous-composant de ce type et d'implémenter la fonction de 
  traitement lié au handler du sous-composant. (voir exemple)
\end{itemize}
%XXXXXXXXXXXXXXXXX
%insérer des exemples de prog
%XXXXXXXXXXXXXXXXX

\section{Propriétés published}\label{sec:published}
Les propriétés placées dans l'objet "published" du composant sont en général utilisées pour
la logique de l'application. Certains "published" sont déjà présents dans les objets Enyo. 
Par exemple, la propriété "content" de l'objet "Control" est elle-même publiée.

Enyo traite ces propriétés comme étant des variables publiques et génère automatiquement : 
getter, setter ainsi qu'une méthode valueChanged appelée à chaque appel du set avec une nouvelle valeur.
\\\medskip

En résumé, un composant déclaré ainsi : 
\begin{JavaScript}
  {kind:"app", published:{maVariable:0}}
\end{JavaScript}
est équivalent à :
\begin{JavaScript}
  {kind:"app", maVariable:0, 
    getMaVariable:function(){..}, 
    setMaVariable:function(nouvelleValeur){..}, 
    maVariableChanged:function(ancienneValeur){}}
\end{JavaScript}

%XXXXXXXXXXXXXXXXX
%insérer des exemples
%XXXXXXXXXXXXXXXXX

Cependant il reste possible d'initialiser les valeurs de propriétés 
déjà publiées hors de l'objet \emph{published} de la même façon dont nous procédions 
avec le champ \emph{content} jusqu'à présent. Cela aura uniquement pour effet de  masquer la première 
initialisation que le constructeur Enyo aura effectué.

Note : la propriété \emph{monObjet.published.maVariable} ne sert qu'à indiquer au 
constructeur d'objet Enyo qu'il doit générer la variable et sa barrière d'abstraction. 
Des effets de bords sur cette variable après instanciation n'auront aucun effet. 

\section{Copie, partage et réutilisation de composants}

Il est possible de réutiliser des composants déjà créés. Il faut cependant faire attention à ne pas
lui attribuer de nom. Dans le cas échéant, le constructeur considèrera que l'objet existe déjà et 
tous les traitements visant une des occurrences de ce nom seront appliqués au premier objet défini.

\begin{JavaScript}
var unBouton = {kind:"Button", name:"bouton1", 
                handlers:{ontap:"clique"}, 
                content:"un bouton",
                clique:function(){
                  this.setContent("Reception du click")
                }
};

enyo.kind({name:"App", components:[unBouton,unBouton]});
\end{JavaScript}

Dans cet exemple, au moment de l'instanciation un message d'avertissement concernant une collision de nom 
est émis mais les deux boutons s'afficheront correctement. Cependant, on constate bien que les clicks 
émis sur l'un ou l'autre des boutons n'auront pour effet de changer que le premier des deux boutons.\\\medskip

Typiquement, si l'on souhaite réutiliser un composant déjà implémenté,
il est recommandé de procéder ainsi :

\begin{JavaScript}
enyo.kind({kind:"Button", name:"MonBouton", 
           handlers:{ontap:"clique"}, 
           content:"un bouton",
           clique:function(){
             this.setContent("Reception du click")
           }
});

enyo.kind({name:"App", 
           components:[
             // Creation de deux MonBouton
             {kind:"MonBouton", name:"bouton1"},
             {kind:"MonBouton", name:"bouton2"}]
});
\end{JavaScript}

Dans ce cas-ci, on commence par décrire un nouvel objet bouton avec son fonctionnement
personnalisé puis en l'étend avec la fonction \emph{enyo.kind} et ainsi pouvoir le
réutiliser dans une nouvelle description de composant.

\section{Hiérarchie des objets}

On peut considérer la plupart objets Enyo dans la hiérarchie comme des noeuds HTML,
en effet, lorsque l'on construit son application, la forme arborescente de l'arbre se traduit
particulièrement bien sous une forme HTML. Ainsi, un \emph{Button} Enyo est lui-même transformé
en un noeud ``<button>'' à l'instanciation. Il est donc aisé de se représenter le rendu final 
d'une application.

Voici un aperçu de l'arbre des composants d'Enyo :\medskip

\begin{tikzpicture}
  [grow=right, level 1/.style={sibling distance=3em},
    level 2/.style={sibling distance=3em}, level distance=2.5cm]

\node (object) {Object}
  child {node (compo) {Component}
    child {node (uicompo) {UIComponent}
      child {node (control) {Control}
        child {node (etc) {...}}
        child {node (input) {Input}
          child {node (textarea) {TextArea}}
          child {node (checkbox) {Checkbox}}
        }
        child {node (Popup) {Popup}}
        child {node (option) {Option}}
        child {node (image) {Image}}
      }
    }
    child {node (animator) {Animator}}
  }
  child {node (async) {Async}
    child {node (ajax) {Ajax}
    }
  };
\end{tikzpicture}

L'arbre complet peut-être trouvé en annexe (\ref{hier-complet}).

La plupart des objets héritent de la classe \emph{Control}. En effet, c'est à partir de celle-ci
que l'on obtient des fonctions d'affichage et la génération d'un noeud html.
Les objets parents de \emph{Control}, tel que \emph{Component} ou \emph{Object} ne servent, en pratique,
qu'à établir une logique dans le code (création de temporisateurs, calculs de valeurs, etc.).
On peut donc tout à fait se passer de ceux-ci lors de l'utilisation en OCaml\footnote{La logique de
l'application étant gérée par le programme OCaml}.

\section{Bibliothèques additionnelles}

Enyo possède un ensemble de bibliothèques supplémentaire apportant du contenu à l'API de base.
On peut par exemple utiliser la bibliothèque Onyx pour fournir une interface graphique 
supérieure ou encore Layout pour organiser ses composants de la manière souhaitée.

Ces objets viennent généralement se greffer à la hiérarchie existante. Par exemple,
``Onyx.Input'' hérite de ``Control''.

Pour pouvoir utiliser cette extension, il est nécessaire de l'indiquer à l'application.
Il faut faire appel à \emph{enyo.depends} sur la liste des bibliothèques à lier.
Par exemple, si l'on souhaite utiliser des objets d'``Onyx'' et de ``Layout'', il faut procéder ainsi :

\begin{JavaScript}
  enyo.depends{ "$lib/Onyx", "$lib/Layout"}
\end{JavaScript}

Il est bien sûr requis de fournir le dossier contenant ces extensions au même niveau que le dossier
``enyo''. Voici une arborescence classique d'une application Enyo :

\begin{tikzpicture}
  [level 1/.style={sibling distance=3cm},
    level 2/.style={sibling distance=1.5cm}]
  \node {application}
    child { node {enyo}
      child{ node {enyo.js}}
    }		
    child { node {lib}
      child{node {Onyx}}
      child{node {Layout}}
      child{node {...}}
    }
    child { node {index.html}}
    child { node {dependances.js}}
    child { node {source}
      child { node {App.js}}
      child { node {App.css}}
    };
\end{tikzpicture}

\subsection{Onyx}

La bibliothèque Onyx enrichit l'application en apportant un ensemble de style prédéfinis pour 
des objets de bases. L'objectif étant d'obtenir une interface graphique correcte sans avoir
à passer du temps sur le style de l'application.

\subsection{Canvas}

Cette extension gère la partie canvas d'HTML5 d'Enyo. Elle définit des objets pour simplifier 
le code JavaScript et ainsi s'abstraire légèrement de la couche basse d'HTML5.

Cependant, cette bibliothèque a ses limites et l'intérêt décroit à mesure que l'application se complexifie.
Il est parfois inévitable de recourir aux fonctions de bases de manipulation de canvas pour obtenir
le résultat escompté.

\subsection{Layout}

Permet de disposer les composants de l'application avec flexibilité.
L'usage de cette bibliothèque n'est malheureusement pas très intuitive, elle nécessite 
exemples et lecture de la documentation présente sur le site d'Enyo.

%%%%%%%%%%%%%%%%%%
%   CHAPITRE 3   %
%%%%%%%%%%%%%%%%%%
\chapter{Interfaçage d'Enyo en OCaml : Nouveau plan}\label{chap:model}

\section{Interopérabilité entre JavaScript et OCaml}

Pour une adaptation en OCaml d'Enyo, il est nécessaire d'établir une communication
entre les deux langages. Cela présente une première difficulté étant donné le peu
de points communs existants. Deux projets ont été développés sur le sujet : Le premier 
``OBrowser''\cite{pwt}, implémente une machine virtuelle interprétant le bytecode OCaml
en JavaScript au runtime. Le second ``Js\_of\_ocaml'', compile le bytecode pour générer
un fichier Javascript qui pourra être ajouté dans une page web.

Le problème principal des deux projets réside dans la représentation des valeurs
d'un monde à l'autre. Pour chacun des cas, il a fallu trouver un moyen efficace de
pouvoir transmettre et interpréter des valeurs tout en assurant le typage.
Cela est réalisé par l'analyse du bytecode qui en redéfinissant les blocs OCaml
en valeur JavaScript permettent leur utilisation dans ce monde-ci et inversement pour
les valeurs JavaScript traduies en blocs. Fournissant une sur-couche d'API JavaScript,
le programmeur OCaml est ainsi à même d'utiliser des fonctions JavaScript au sein de son
programme en pouvant s'assurer du typage. L'utilisation de fonctions OCaml en JavaScript
est également possible.

Le problème de l'interopérabilité avec Enyo suit les mêmes contraintes :
Le dynamisme de JavaScript empêche une traduction directe. Il est donc intéressant
d'utiliser l'un de ces outils afin de nous fournir une couche haute pour l'implémentation de
ces valeurs.


\section{Js\_of\_ocaml : Présentation des outils}

Le choix s'est porté sur Js\_of\_ocaml. Celui-ci fournit une représentation assez complète des
objets JavaScript et une manipulation aisée des champs. Une telle liberté peut cependant être source
d'erreurs, venant en particulier du fait que nous utilisons certaines fonctions non-typés.
L'intérêt se limite ainsi à l'interopérabilité et nous nous assurerons du typage lors de la définition
du modèle.

Nous utiliserons en particulier le module Js.Unsafe qui contient qui contient une couche assez basse
de l'API ``Js\_of\_ocaml'' pour nous permettre des libertés, en particulier nous 
nous intéresserons aux fonctions suivantes :

\begin{OCaml}
(*Type haut permettant de contenir l'ensemble des valeurs JavaScript possibles*)
type any

(*Fonction d'accession aux variables JavaScript
  => variable "monobj", renvoie l'objet "monobj" défini dans l'environnement JavaScript*)
val variable : string -> 'a

//Coercition d'une valeur vers le type haut
val inject : 'a -> any

(*Fonctions d'accession et de modification de champ des objets JavaScript*)
val get : 'a -> 'b -> 'c
val set : 'a -> 'b -> 'c -> unit

(*Appel de fonction JavaScript avec un tableau d'argument
  => fun_call (variable "alert") [| inject 42 |]
  = affiche 42*)
val fun_call : 'a -> any array -> 'b

(*Appel d'une méthode d'un objet JavaScript
  => let unsafe_str = meth_call unObj "toString" [| |]*)
val meth_call : 'a -> string -> any array -> 'b

(*Création d'un objet JavaScript*)
val new_obj : 'a -> any array -> 'b
\end{OCaml}

Cette liberté a évidemment un prix. Ces fonctions sont très dangereuses et c'est pourquoi il faut 
faire preuve de beaucoup de précautions à l'utilisation pour éviter les erreurs que pourrait engendrer
l'interfaçage.
Notamment lorsque l'on passe des arguments aux appels de méthodes, il est nécessaire des les transcrire
en valeurs JavaScript et inversements pour récupérer les valeurs retournées.

La traduction de ces valeurs s'opérent grâce aux fonctions de traductions étant elles typées.
\begin{OCaml}
  val bool : bool -> bool t
  val to_bool : bool t -> bool
  val string : string -> string t
  val to_string : string t -> string
  val array : 'a array -> 'a js_array t
  val to_array : 'a js_array t -> 'a array
  (*...*)
\end{OCaml}

Il faut donc s'assurer de la sécurité des appels des fonctions ou méthodes, par exemple,
pour ne pas générer d'erreurs lors de l'ajout de propriétés dans un objet. Ainsi, 
si l'on veut ajouter un champ de type string dans un objet JavaScript, il faut
faire appel à la fonction \emph{string} qui se chargera de l'instanciation d'un objet \emph{String}
JavaScript et donc de toutes ses méthodes.

\begin{OCaml}
  set unObj "couleur" (string "rouge")
\end{OCaml}


\section{Typage OCaml d'Enyo}
\subsection{Typage direct}
%correspondance des types OCaml pour Enyo
\subsection{Types fantômes}
\subsection{Propriété de sous-typage}

\section{Description en OCaml des déclarations d'Enyo}
\subsection{Modèlisation}
\subsection{Propriétés}
%ajout d'options 
\subsection{\'Evénements}

\section{Génération et Instanciation}
\subsection{Procédure}
%\subsection{\'Evénements}
\subsection{Méthodes}
\subsection{Introspection des objets}


%%%%%
\chapter{Interfaçage d'Enyo en OCaml}

Pour une adaptation en OCaml d'Enyo, il est nécessaire d'établir une communication
entre les deux langages. Cela présente une première difficulté étant donné le peu
de points communs existants. Nous pouvons néanmoins résoudre le problème en
utilisant  \emph{Js\_of\_ocaml}, un compileur de byte-code OCaml vers Javascript développé par 
l'équipe d'\emph{Ocsigen}. Nous obtenons ainsi un moyen fiable d'assurer une traduction, 
de déclarer des objets Javascript ou encore d'attacher des méthodes à ces derniers, le tout
en restant dans le monde OCaml. La translation assurée, nous pouvons nous concentrer sur 
la modélisation d'Enyo.

Le point critique de l'interfaçage se situe dans l'utilisation des traits 
dynamiques de Javascript, par exemple, utiliser la méthode \emph{setContent(value)} sur un
\emph{kind:"Control"} qui ne possèdera cette méthode qu'après instanciation est
rédibitoire pour une transcription directe en OCaml.\medskip

Après considérations, j'ai choisi de m'arrêter sur un modèle reposant sur une différenciation
d'un objet en construction de celui d'un objet instancié. On représente ainsi un objet \emph{Enyo}
en OCaml grâce aux propriétés le définissant.\medskip

\begin{tikzpicture}[node distance=7cm, auto,]
  \node[punkt] (kind) {Description de l'objet};
  \node[punkt2,right=of kind] (jsobj) {Objet Javascript}
  edge[pil, <-] node[auto]  {Génération et Instanciation} (kind.east);
\end{tikzpicture}
\medskip

L'extension de l'objet en cours de création avec propriétés et méthodes personnalisées est 
abandonné. Cet aspect d'Enyo est nécessaire pour la partie logique de l'application, ce dont 
nous pouvons nous abstraire puisque nous souhaitons un modèle de calcul reposant sur 
un programme OCaml classique.

A partir de là, les objets de l'interface de programmation (API) d'Enyo et leurs méthodes 
sont suffisants si l'on considère utiliser Enyo comme une interface graphique multi-plateforme
pour OCaml. Il est également nécessaire de pouvoir traiter les événements utilisateurs pour bénificier 
d'une interface dynamique et réactive.

Dans un premier temps (\ref{repobj}), nous présenterons la structure choisie, son fonctionnement et les raisons
de cette décision.

Ensuite, la définition des propriétés et leur implémentation (\ref{gestprop})

Enfin, nous nous attarderons sur les événements (\ref{gestevent}).

\section{Représentation des objets}\label{repobj}
\subsection{Modélisation}
D'après le modèle choisi, on distingue donc la représentation en deux types. Le premier,
à créer par l'utilisateur, que nous nommerons ``kind'' par la suite. Et le second,
résultant d'une instanciation, possédant toutes les méthodes décrit dans l'API Enyo, que
nous appellerons ``obj''.

Dans un soucis d'aisance, j'ai fait le choix d'annoter ces ``objets'' à l'aide de types fantômes.
Associés aux variants polymorphes, on profite ainsi d'un sous-typage.
Cela a pour avantage de permettre un partage des noms de méthodes appartenant à plusieurs objets. 
Le désavantage majeur de cette solution reste cependant la confusion liée 
à la génération des messages d'erreurs lors du développement d'une application.

Nous obtenons ainsi les deux types ``kind'' et ``obj'' qui seront utilisés avec l'ensemble de tous 
les type objets Enyo en précisant l'utilisation covariante du paramètre.
\begin{OCaml}
  type any_id = 
        [ `CONTROL | `INPUT | `BUTTON (* | ... *) ]
  type +'a kind (* constraint 'a = [< any_id] *)
  type +'a obj
\end{OCaml}
Les signatures de méthodes partagées prennent alors la forme suivante :
\begin{OCaml}
  sig:
  (*...*)
  
     val setContent : 
         [< `BUTTON | `CONTROL (* ... *) ] obj 
         -> string 
         -> unit
  
  (*...*)
  end
\end{OCaml}

Pour construire l'objet Enyo, ce dernier doit pouvoir contenir:
\begin{itemize}
\item Les propriétés de bases incluent dans l'API Enyo
\item Les fonctions de traitement des différents évenements
\item La liste des sous-composants
\end{itemize}

Le choix s'est donc porté sur une structure de ce type :

\begin{OCaml}
  type handler = Handler of string * (any_id obj -> any_id obj -> any_event obj -> bool)
  type js_value = Int of int | String of string 
                | (*...*) 
                | Array of js_value list 
                | Component of any_id obj 
  type +'a kind = {id:string; 
                   components: any_id kind list; 
                   handler_list:handler list; 
                   prop_list : (string * js_value) list 
                  }
\end{OCaml}

J'incorpore à la structure un champ \emph{id} pour pouvoir permettre une introspection 
sur les futurs objets instanciés (voir \ref{gestevent}).

Quant à la représentation Javascript, elle est simplement représentée par un objet Javascript
du module ``Js.Unsafe'' de la bibliothèque \emph{js\_of\_ocaml}.

\begin{OCaml}
  type +'a obj = Js.Unsafe.any
\end{OCaml}

Cela permet un travail direct sur les différents appels à transmettre au monde Javascript.

Munis des structures, on peut désormais s'intéresser à la construction proprement dite de cette
représentation OCaml d'un objet Enyo.

\subsection{Création}

Pour cette phase de création, j'ai décidé de m'inspirer du module Tk d'OCaml qui utilise les options
pour les fonctions prenant un nombre important d'arguments omissibles.

Plusieurs autres solutions auraient été possible, telle qu'une liste d'association mais dont
je n'ai pas su trouver de façon élégante pour permettre à l'utilisateur de l'employer aisément.

Ainsi, nous obtenons un constructeur, ici de bouton pour l'exemple, de cette forme :

\begin{OCaml}
  val button:
    ?components:any_id kind list
    -> ?content:string
    -> ?showing:bool
    -> ?src:string
    -> ?canGenerate:bool
    (* ... *)
    -> ?ontap:([`BUTTON] obj -> 
               any_id obj -> 
               [`GESTURE] obj -> bool)
    -> unit -> [>`BUTTON] kind
\end{OCaml}

donnant à l'utilisation, pour un bouton contenant un simple label, ceci :

\label{bouton_kind_ex}
\begin{OCaml}
  let mon_bouton = bouton ~content:"Valider" ()
\end{OCaml}

Quelques bémols, cependant, à cette manière de procéder :
\begin{itemize}
\item Le nombre d'arguments de chaque constructeur augmente au fur et à mesure que l'objet
soit profondément situé la hiérarchie,
\item Les messages d'erreurs générés deviennent rapidement indigestes,
\item Une grande redondance parmi les différents constructeurs.
\end{itemize}

Finalement, le code du constructeur revient à inclure chaque argument optionnel passé dans différentes 
listes (typiquement, la liste de propriétés et celle de fonctions de traitement) en testant
au préalable leur existence.
En ajoutant la liste de composants quelques constantes, un ``ID'' pour l'introspection et 
le champ ``kind'' nécéssaire à Enyo, l'objet est prêt à être instancié en tant qu'application 
ou à être passer en tant que sous-élément dans un autre objet par sa liste de ``components''.

\begin{OCaml}
  let button
      ?(components=[])
      ?content
      ?ontap
      (*...*)
      () =
    let prop_list= ref [("kind", String "Button")]
    and handler_list= ref [] in
    (match content with 
        Some v -> prop_list := ("content",String v)
                               ::!prop_list 
        | None -> ());
    (match showing with 
        Some v -> prop_list := ("showing",Bool v)
                               ::!prop_list 
        | None -> ());
    (*...*)
    (match ontap with 
        Some v -> handler_list := Handler ("ontap",v)
                                  ::!handler_list 
        | None -> ());
    {id="BUTTON"; 
     components=components;
     prop_list=(!prop_list);
     handler_list=(!handler_list)}
\end{OCaml}

\subsection{Instanciation}

Après avoir obtenu notre représentation OCaml, il est nécessaire de le traduire en Javascript.
C'est le rôle de la fonction \emph{Instanciate}.

Cette dernière parcourt les différentes listes (propriétés, évenements, et composants) et construit
un objet Javascript formé selon les spécifications d'Enyo que nous passons à \emph{enyo.kind} et 
retournant l'instance de celui-ci avec un nom par défaut si non-fourni.

Une application d'\emph{instanciate} sur le ``bouton kind'' définit plus haut (\ref{bouton_kind_ex}) 
équivaut en Javascript alors à :

\begin{JavaScript}
  enyo.kind({kind:"Button", name:"Default_name", content:"Valider", _onyo_id:"BUTTON"});
  var obj_enyo = new Default_name();
  return obj_enyo;
\end{JavaScript}

La fonction étant relativement indigeste, je détaillerai ici les manipulations éxecutées :

\begin{enumerate}
\item La liste de propriétés est évaluée et chaque type de valeur est traduite si besoin en Javascript.
  Chaque propriété est ensuite injectée dans un nouvel objet Javascript à l'aide de \emph{Js.Unsafe.set}
\item \label{wrap_meth} 
  Les différentes fonctions de traitement présentes sont ``enveloppées'' dans l'objet en tant que méthodes
  dans l'objet avec un appel à \\\emph{Js.wrap\_meth\_callback} prenant une fonction dont le premier argument
  est l'objet lui-même représentant le \emph{this}. Il est aussi nécessaire de déclarer l'objet
  \emph{handlers} pour assurer le traitement des événements.
  (Plus de détails : \ref{gestevent}).
\item Récursion sur tous les sous-composants présents et transformation de la liste de ces derniers en 
  un tableau Javascript à passer au champ ``components'' de l'objet.
\end{enumerate}

Voici, une forme épurée de cette fonction :

\begin{OCaml}
    let instanciate (kind:([< any_id] as 'a) kind) : 'a obj =
      let rec build_component_tree : 'a. ([< any_id] as 'a) kind -> Js.Unsafe.any = fun kind ->
          let js_obj = Js.Unsafe.new_obj (variable "Object") [||]
          in
          (* 1 *)
          ;
          (if kind.handler_list != [] then
              (* 2 *)
          );
          (if kind.components != []  then
              (* 3 *)
              let array_component = Array.of_list (List.map build_component_tree kind.components) in
	      Js.Unsafe.set js_obj "components" (array array_component));
          Js.Unsafe.set js_obj "_onyo_id" (string (kind.id));
          js_obj in
       kind_it (build_component_tree kind)
\end{OCaml}

%Note : il a été nécessaire d'utiliser un quantificateur universel pour unifier la récursion 
%avec la fonction afin d'assurer le polymorphisme\\\medskip

Enfin, on passe cet objet à la fonction \emph{enyo.kind} pour récupérer un objet Enyo
dont les méthodes sont désormais appelables tel que \emph{setContent} si l'objet hérite 
de \emph{Control} dans la hiérarchie Enyo.

L'utilisateur est désormais en mesure d'afficher son application en effectuant un appel
à la méthode \emph{renderInto} (si l'objet en possède la méthode) sur l'objet instancié.

\section{Gestion des propriétés et des méthodes}\label{gestprop}

Pour pouvoir représenter les valeurs Javascript faiblement typées, il a été nécessaire de définir un 
ensemble de ces type valeurs. Un type somme est suffisant :

\begin{OCaml}
    type js_value = Int of int | String of string 
                  | Char of char | Float of float 
                  | Dom_node of dom_node | Bool of bool 
                  | Array of js_value list | Component of any_id obj 
\end{OCaml}

Note: j'ai trouvé préférable de représenter le tableau Javascript sous forme de liste,
le premier étant plus proche de la liste OCaml que du tableau.\medskip

Les propriétés possédés par les objets Enyo sont implémentées en tant que ``Published'' 
(voir \ref{sec:published}). Ces propriétés possèdent donc des accesseurs (get) et des modificateurs (set) ainsi
que la fonction \emph{<valuename>Changed}. Il faut alors pour chaque propriété générer ces méthodes.

J'ai fait le choix d'omettre la possibilité de définir un \emph{<valuename>Changed} qui peut cependant,
si la logique de l'application le nécessite, être remplacé par un simple appel de fonction au moment
de la modification de propriété.

Au niveau de l'implémentation, j'ai ajouté ces accesseurs/modificateurs pour chaque propriété définie 
et appelable par tous les composants en héritant.

Afin d'assurer une bonne traduction Javascript/OCaml, il a fallu selon les types employés traduire
les variables. Pour cela, l'API de ``js\_of\_ocaml'' propose les fonctions nécessaires :

\begin{OCaml}
  val bool : bool -> bool t
  val to_bool : bool t -> bool
  (*...*)
  val array : 'a array -> 'a js_array t
  val to_array : 'a js_array t -> 'a array
\end{OCaml}

Selon le type de la propriété, on s'assure alors d'effectuer correctement cette traduction :

\begin{OCaml}
  let getContent this () =
    let value = Js.Unsafe.meth_call 
                     this 
                     "getContent" 
                     [||] in (* Appel de la methode sur l'objet Javascript *)
    to_string value (* conversion de la valeur js obtenu en valeur caml *)
\end{OCaml}
\clearpage %A vérifier plus tard
\begin{OCaml}
  let setContent this chaine1 =
    let _ = Js.Unsafe.meth_call 
                this 
                "setContent" 
                [|Js.Unsafe.inject (string chaine1)|] in (* conversion caml -> javascript *)
    ()
\end{OCaml}

Les propriétés Enyo étant pour la plupart initialisées par défaut, je n'ai pas jugé urgent
de gérer les cas de valeur ``null''. Il faudrait, pour certains cas, fournir cette sécurité.\medskip

Une autre difficulté s'est posée concernant les méthodes polymorphiques. Une méthode de type
\emph{setProperty(name, value)} est difficile à representer en OCaml, c'est pourquoi j'ai préferé
laisser ce problème de côté afin de ne pas compliquer le typage.

Quelques propriétés ont ainsi été laissées de côté, comme par exemple celles prenant des
objets Javascript non-Enyo en valeur.

\begin{JavaScript}
//bounds : {left: _offsetLeft_, top: _offsetTop_, width: _offsetWidth_, height: _offsetHeight_}
this.setBounds({width: 100, height: 100}, "px");
this.setBounds({left:"100", top:"40", width: "10em", right: "30pt"});
\end{JavaScript}

Il conviendrait d'étudier chaque cas particulier et de fournir une équivalence sûre quitte à
réduire les possibilités d'usage. Par exemple, fournir une fonction à quatre arguments qui
n'agit sur une seule unité de mesure.\medskip

Enfin, certaines propriétés et arguments de méthodes peuvent changer de type selon leur usage 
et il n'est pas toujours agréable d'avoir à effectuer une conversion OCaml (ex: \emph{string\_of\_int})
sur ces appels. Une solution pourrait être de dupliquer ces méthodes, par exemple, en les préfixant 
par le type. Ainsi, la propriété ``value''  de l'objet Enyo \emph{Slideable} prenant un int, qui par ailleurs 
effectue une collision de nom avec celle de l'objet \emph{Input} (voir \ref{chap:idl}), pourrait 
devenir \emph{int\_getInput} s'appliquant à l'objet Slideable et à ses descendants, tandis que \emph{Input}
posséderait une méthode \emph{string\_getInput}.

\section{Gestion des événements}\label{gestevent}
\subsection{Fonctionnement}
Dans Enyo, la spécification des événements et de leur fonctions de traitement gardent toujours la même
forme.

\begin{JavaScript}
  fonctionTraitement: function(emetteur, evenement){
                          //traitement
                          return bool; }
\end{JavaScript}

La fonction attachée prend en paramètres l'émetteur de cet événement, c'est-à-dire, 
le composant qui a genéré ou propagé celui-ci, et l'objet événement lui-même.
Celui-ci retourne un booléen spécifiant si la propagation doit être effectué.
(voir : \ref{sec:event} pour plus de détails)

Pour adapter l'évenement en OCaml, j'ai choisi ce type :
\begin{OCaml}
  type any_event = [`GESTURE (* |  Ensemble des evenements *) ]
  type handler = Handler of string * (any_id obj -> any_id obj -> any_event obj -> bool)
\end{OCaml}

La gestion de l'événement est donc représentée par un couple contenant le nom de l'événement et la 
fonction de traitement associée avec en premier argument le \emph{this} de la méthode (voir : \ref{wrap_meth}).

En informant la signature dans chaque constructeur, il devient aisé de constituer un typage correct.
Ainsi, nous pouvons résoudre le \emph{this} selon l'objet en cours de construction et l'événement attendu.
L'émetteur ne peut cependant pas être résolu, nous y reviendrons plus loin.

Pour un bouton, nous attendrons un argument de ce type :

\begin{OCaml}
  val button:
    (*...*)
    -> ?ontap:([`BUTTON] obj -> any_id obj -> [`GESTURE] obj -> bool)
    -> unit -> [>`BUTTON] kind
\end{OCaml}

A l'utilisation, nous pourrons alors utiliser les méthodes de button sans causer de conflit de type 
puisqu'employé dans un contexte correct :

\begin{OCaml}
let traitement_bouton this emetteur evenement = 
   setContent this "Nouveau contenu";
   setDisabled this true;
   true

let mon_bouton = button ~content:"contenu" ~ontap:traitement_bouton ()
\end{OCaml}

J'ai défini les événements de la même manière que les objets. Il n'est, en revanche, pas nécessaire 
de partager les méthodes appliquables sur les  événements puisqu'ils ne sont pas liés entre eux. Il 
est cependant nécessaire d'affecter à chaque type d'événements des fonctions d'accesseurs pour 
récupérer les informations contenus.

Par exemple, pour l'événement souris, rebaptisé ``gesture'' pour se conformer à Enyo, les accesseurs,
stipulés dans la définition IDL W3C, sont définis :

\begin{OCaml}
  val gesture_screenX : [`GESTURE] obj-> int
  val gesture_screenY : [`GESTURE] obj-> int
  val gesture_identifier : [`GESTURE] obj-> int
  val gesture_detail : [`GESTURE] obj-> int
  (* ... *)
\end{OCaml}
\medskip

\'A l'instanciation d'un objet pourvu d'un événement, il est nécessaire de passer l'événement traité
dans un objet contenu dans le champ \emph{handlers} de celui-ci. Ainsi, le \emph{mon\_bouton} définit plus haut,
aura cette forme en Javascript :

\begin{JavaScript}
  {kind:"Button", handlers:{ontap:"tap"}, 
    tap:function(){ return callback_caml(this) /* gere par js_of_ocaml */ }}
\end{JavaScript}

Cette stratégie permet à chaque composant de posséder sa fonction de traitement mais présente 
un gros soucis lors du ``bubbling'' (remontée d'événement). En effet, il faut pouvoir connaître l'origine
de l'événement à tout moment .

Prenons une situation d'exemple en Enyo avec notre représentation:
\begin{JavaScript}
enyo.kind({
    kind:"Control",
    name:"App",
    components:[
      {kind:"Control",
        components:[
          {kind:"Button", content:"Vert",         
            handlers:{ontap:"tap"},
            tap:function(emetteur, event){
              //emetteur = this 
              return false; //propagation au noeud-parent
            }
          },
          {kind:"Button", content:"Bleu",         
            handlers:{ontap:"tap"},
            tap:function(emetteur, event){
              //emetteur = this
              return false; //propagation au noeud-parent
            }
          }
        ]
        }
      ],
    handlers:{ontap:"tap"},
    tap:function(emetteur, event){
      //emetteur = sous-control
      // change le fond en noir si l'emetteur est b1
      // ou bien en blanc si b2 
      alert(emetteur);
      return true;
    }
})
\end{JavaScript}

Ici, nous voulons, si l'on clique sur l'un des deux boutons, changer la couleur du fond du
``control'' racine selon le bouton cliqué.

Il faut donc pouvoir tester au niveau racine, quelle a été l'origine des événements.
Le problème ici provient de la propagation au composant parent par les boutons.
En effet, l'argument \emph{emetteur} reçu par la racine ne contient pas l'un des deux boutons
mais le ``Control'' intermédiaire par lequel l'événement a transité.

En Enyo, un simple : \emph{emetteur.originator} renverrait le bouton ayant généré l'événement,
et nous permettrait de déterminer la source mais l'implémentation de cette propriété aurait 
peu de sens dans la représentation actuelle puisqu'il impliquerait la présence d'une propriété
originator dans tous les objets manipulé, même en dehors d'un événement.

Une solution à ce problème peut être de conserver, dans son application, une valeur référencé 
contenant l'objet ou l'information nécessaire pour effectuer le traitement souhaité.
Je propose une autre stratégie plus loin : \ref{sub:event_amel}.

\subsection{Introspection}

L'implémentation actuelle de cette stratégie de traitement pose un problème pour typer l'emetteur.
Pour pallier à celà, j'ai fourni lors de la construction des objets un champ supplémentaire
\footnote{Le champ kind étant effacé à l'instanciation Enyo, il n'est pas possible d'en faire usage} à chaque
``kind'' qui va s'étendre à l'objet Javascript : \emph{\_onyo\_id}.

Munis de cette propriété, nous pouvons désormais accéder à ce champ pour le typer correctement.
C'est le rôle de la fonction \emph{as\_a}:

\begin{OCaml}
  val as_a : ([< any_id] as 'a) -> [< any_id] obj -> 'a obj
\end{OCaml}

Si l'objet n'est pas de l'``id'' spécifié, une exception est levée.

Avec ceci, nous pouvons agir sur l'emetteur :
\begin{OCaml}
  let mon_bouton = ~content:"Bouton" ~tap:(fun _ _ _ -> false) ()

  let traitement this emetteur evenement =
     try 
         let bouton_emetteur = as_a `BUTTON emetteur in
         (* bouton_emetteur : [`BUTTON] obj *)
         setDisabled bouton_emetteur true;
         true
     with Bad_kind -> true

  let mon_control = ~components:[mon_bouton] ~ontap:traitement ()
\end{OCaml}

Dans cet exemple, le bouton capture l'événement \emph{ontap} mais le délégue
à son composant racine. Ce dernier essaye de transtyper l'émetteur en bouton
et le désactive.

Cette solution reste cependant inélégante et assez contraignante pour le développeur.

\subsection{Autre stratégie de propagation}\label{sub:event_amel}

Dans un développement Enyo, il est courant de ne définir les fonctions de traitement qu'à la racine de l'arbre
de composants. Ceci permet d'avoir un accès à tous les sous-composants grâce à la table de hash que possède
le composant principal. Nous obtenons ainsi le programme reprenant l'exemple posant problème :
\begin{JavaScript}
  enyo.kind({
    kind:"Control",
    name:"App",
    components:[
      {
        kind:"Control",
        components:[
          {kind:"Button", name:"b1", content:"Vert",
            ontap:"tap"
          },
          {kind:"Button", name:"b2", content:"Bleu",         
            ontap:"tap"
          }
        ]
      }
    ],
    tap:function(emetteur, event){
      if (this.$.b1 == emetteur){
        // couleur noire
      } else if (this.$.b2 == emetteur){
        // couleur blanche 
      }
      return true;
    }
    //Il peut aussi etre interessant de definir une fonction pour chaque bouton :
    tap_b1:function(em, ev){ 
      //couleur noire
      return true;
    }
    tap_b2:function(em,ev){
      //couleur blanche
      return true;
    }
})
\end{JavaScript}

Un comportement semblable pourrait être implémenté résolvant par la même occasion, le typage de l'émetteur.

Plutôt que de définir les fonctions de traitement au niveau des objets, on pourrait les définir
sous des labels uniques, à la racine. Nous obtiendrions, une signature dans le constructeur
pour le traitement de ce type ressemblant à :
\begin{OCaml}
  button ?content:string
         (* ... *)
         ?ontap:([`APPLICATION] obj -> 
                 [`BUTTON] obj -> 
                 [`GESTURE] obj -> bool)
\end{OCaml}

Il conviendrait à priori de définir un nouvel objet ``Application'' sur lequel il serait possible de
fournir une lecture de la table de hash et ainsi d'agir sur les différents objets.

L'instanciation à même titre retournerait un : \emph{[`APPLICATION] obj}.

J'essayerai de convenir de la validité d'une telle approche et de son implémentation dans une 
version future.

%%%%%%%%%%%%%%%%%%
%   CHAPITRE 4   %
%%%%%%%%%%%%%%%%%%

\chapter{Langage de Description d'Interface (IDL)}\label{chap:idl}

Enyo fournit une API suffisamment grande (une centaine d'objet environ) pour qu'il devienne 
rébarbatif d'implémenter ceci à la main. L'implémentation d'un IDL s'est donc révélé être 
une bonne option pour pouvoir représenter facilement la hiéarchie de composants.

OCaml étant un outil tout indiqué pour ce type de travail, je n'ai pas eu à faire de parallèle 
avec d'autres langages (mis à part quelques scripts Perl pour parser la hiérarchie).

En premier lieu, il convient de présenter les structures choisies pour abriter les objets.
Puis, j'expliquerai brièvement la génération du code et les problèmes rencontrés.

\section{Interface des objets}

Un objet Enyo présente 3 grandes parties distinctes :
\begin{itemize}
  \item Les méthodes,
  \item les propriétés,
  \item les événements qu'il est capable de ``capturer''.
\end{itemize}

Pour les méthodes et les propriétés, il est nécessaire de fournir le type à l'IDL pour assurer à
OCaml la sûreté du typage.

Les événements, quant à eux, possèdent un lien vers le type d'événement et les champs qu'il faut préciser.

\begin{OCaml}
type values_rep =  String | Int | Float | Bool | Unit 
		   | Component | Array of values_rep | Dom_node

type method_rep = Method of string * values_rep list

type attributes_rep = Attribute of string * values_rep * bool (* bool => Generate propChanged function? *)

type event_rep = Event of string * attributes_rep list

type handler_rep = Handler of string * event_rep

type object_rep = Type of string * method_rep list * attributes_rep list * handler_rep list
		  | Type_gen of object_rep * string * method_rep list * attributes_rep list * handler_rep list
                  (* Le deuxieme element de la somme n'a plus de reel utilite, mais n'ayant pas eu le temps
                  de retravailler le code, il est a ignorer *)
\end{OCaml}

J'ai donc choisi de représenter l'objet Enyo par :
\begin{itemize}
\item Son nom Enyo,
\item une liste de représentation de méthodes,
\item une liste de représentation de propriétés\footnote{L'utilisation du booléen pour savoir si 
une fonction de type <value>Changed doit être implémenté est temporairement abandonné, pour
les raisons spécifiées dans la partie de gestion des propriétés \ref{gestprop}},
\item une liste des noms d'événements que ce type d'objet est capable de traiter.
\end{itemize}

On obtient ainsi pour l'objet ``Control'' d'Enyo, la représentation suivante :

\begin{OCaml}
Type ("Control",
      [
	Method ("render", [Unit; Unit]);
	Method ("rendered", [Unit; Unit]);
	Method ("renderInto", [Dom_node; Unit]);
	Method ("setBounds", [Int; Int; Int; Int; Unit]);
	Method ("show", [Bool; Unit]);
	Method ("write", [Unit; Unit]);
        (*...*)
      ],
      [
	Attribute ("style", String, true);
	Attribute ("content", String, true);
	Attribute ("allowHtml", Bool, false);
	Attribute ("src", String, true);
        (*...*)
      ],
      [Handler ("ontap", Gesture_event._gesture)]
)
\end{OCaml}

En étudiant l'API, j'ai pu transcrire les méthodes et propriétés en m'efforçant de typer au plus juste.
En revanche, quelques libertés ont été prises:
\begin{itemize}
\item Je n'ai pas jugé utile l'ajout de méthodes trop polymorphes telle que : \emph{setProperty(name, value)}.
\item Par manque de documentation de l'API sur certaines, il a été quelques fois nécessaire de compléter 
  moi-même les signatures de méthodes en effectuant des tests en restant toutefois rigoureux.  
\item J'ai omis d'inclure les méthodes protégées, pour la simple raison que nous ne traitons pas
  l'extension objet, ici.
\end{itemize}\medskip

Je complète ainsi chaque objet et l'inclus dans un arbre d'``object\_rep'' recréant ainsi la hiérarchie
Enyo nécessaire permettant à chaque composant d'hériter des méthodes et attributs.
La hiérarchie désormais complète, la génération de code peut débuter.

\section{Générateur de code}

L'objectif de cette génération de code est d'obtenir en sortie un module complet basé sur la représentation
IDL avec le modèle choisi \ref{chap:model}.

La difficulté de cette étape a été le parcours de l'arbre pour résoudre les dépendances:
\begin{itemize}
\item Un objet doit transmettre à ses fils toutes les propriétés et ``handlers'' d'événements 
  qu'il possède afin que leurs constructeurs soient en mesure de les proposer à l'utilisateur
\item En revanche, les méthodes auront besoin de la liste de tous les identifiants de sa descendance pour
  pouvoir spécifier les méthodes et permettre aux enfants de les appeller
\end{itemize}\smallskip

Une simple récursion sur l'arbre permet l'obtention d'un résultat satisfaisant.

Plus tard, j'ai constaté qu'un cas spécial ne permettait pas un parcours aussi banal :
lorsque deux objets définissent des méthodes ou proprietés de même noms sur des branches différentes
de l'arbre, il y a collision de noms et il est impossible de factoriser simplement les dépendances
\footnote{J'omets le fait que deux méthodes de même nom peuvent être de types différents, 
ce qu'il faudrait traiter si le cas était présent}.

Afin de résoudre ceci, je réalise des parcours préalables en remplissant des tables de hachage contenant
en clé le nom de méthodes et de proprietés avec en valeur leurs listes des dépendances.
J'applique ensuite la récursion, en consultant la table de hachage pour résoudre les incohérences.

Note : On aurait également pu ``tagger'' les différentes incohérences au moment de la représentation de l'objet
en spécifiant que cette méthode ou propriété est partagée avec un autre objet. Plusieurs solutions sont
possibles...\medskip

L'arbre de résolutions de dépendances construit, il ne reste plus qu'à réaliser les impressions
pour pouvoir générer le module OCaml prêt à l'emploi.

%%%%%%%%%%%%%%%%%%
%   CHAPITRE 5   %
%%%%%%%%%%%%%%%%%%


\chapter{Conclusion}

%Le peu de similitude entre JavaScript et OCaml peut présenter une des difficultés majeure dans cet interfaçage.
%Il est alors important de se demander à quel ordre l'utilisation de traits intrusifs est faite dans 
%l'implémentation d'applications Enyo.
%Tout en étudiant cette question, il faut pouvoir réflechir à quel type de représentation de modèle
%ce style s'adapte-t-il le mieux.
%Les objets Enyo étant nombreux, il peut-être important de se donner les moyens de les représenter 
%de façon efficace. La définition d'un IDL peut-elle être un atout pour réaliser ceci?
%
%Premièrement, nous étudierons Enyo pour nous permettre d'obtenir les informations nécessaires à 
%la réflexion d'une approche.
%Ensuite, nous tenterons de fournir un modèle en détaillant les raisons de ce choix et les problèmes
%qui en ressortent.
%Enfin, nous présenterons un IDL permettant de modéliser les objets Enyo.

A travers ce projet, nous avons pu établir un certain nombre de choses.
Tout d'abord, nous avons pu établir que le style de programmation d'Enyo nous offrait
un nombre important de possibilités. Il a fallu étudier ce style, son fonctionnement et 
en abstraire les concepts afin de définir un modèle intéressant et cohérent.

La scission de la représentation et de l'instanciation de l'objet Enyo sur lequel j'ai basé mon modèle
répondent à l'objectif de sûreté que nous nous étions fixé tout en assurant un degré important
de liberté vis-à-vis de l'API d'Enyo.

Enfin, la définition d'un IDL a permis une génération fiable de l'implémentation finale mais aussi de 
permettre de tenir à jour l'API Enyo de manière assez aisée en cas de besoin.

Cependant, le rendu final subit quelques défauts : 
\begin{itemize}
\item L'utilisation de variants, bien que permettant 
  une bonne modélisation, peut être perçue comme trop complexe pour une prise en main rapide. Cela fournit
  également des messages d'erreurs trop aggressifs. 
  Cette dernière remarque s'applique aussi pour l'utilisation des paramètres optionnels lors de la
  construction des objets.
\item L'accession aux différents composants instanciés lors de la définition du traitement 
  des événements n'est pas satisfaisante et gagnerait à être améliorée.
\item L'IDL ne traite pas certains cas particuliers et reste relativement primaire.    
\end{itemize}\medskip

Pour permettre une continuation du projet, il peut-être intéressant de reprendre les problèmes
énoncés avec, par exemple, les différentes approches énoncées dans les différentes sections. Pour résumer :
\begin{itemize}
\item Réfléchir à une représentation différentes que les variants, quitte à rendre le
  module plus lourd, par une approche objet ou type simple.
\item Pouvoir accéder aux instanciations Enyo par le ``kind'' de l'objet, en s'assurant
  de son existence. Par exemple, ainsi :
  \begin{OCaml}
    let bouton_obj = getObj button_kind (* 'a kind -> 'a obj *)
    setContent bouton_obj "Nouvelle valeur"
  \end{OCaml}
\item Il serait bien de compléter la hiérarchie en fournissant des solutions adéquates
  aux cas particuliers. Ensuite, pouvoir fournir des possibilités d'extension à l'IDL en permettant 
  des définitions internes de composants. Par exemple, être capable  de créer un nouvel objet ``ChampValidation'' 
  possédant deux composants ``Input'' et ``Button'' ainsi qu'une fonction de validation.
\item Il faudrait réfléchir à une représentation des noeuds du DOM. Est-ce que l'on souhaite passer 
  par ``js\_of\_ocaml'', fournir une chaine le représentant, ou encore définir quelques noeuds principaux
  et brider leur utilisation?
\end{itemize}\medskip

Sur le plan personnel, ce stage m'a été, je pense, très profitable ; j'ai pu consolider mes bases en OCaml
ainsi que mon engouement pour ce langage. Je le dois en bonne partie à l'équipe capable et dynamique
dont j'ai eu la chance d'être entouré pendant ces deux mois.
De plus, cette expérience au sein d'un laboratoire m'a sérieusement fait réfléchir à 
une éventuelle formation vers la recherche.
Je me satisfait donc d'avoir pu effectuer ce stage dont j'en sors enrichi.

\chapter{Annexes}

\section{Hiérarchie complète Enyo}\label{hier-complet}

\lstinputlisting[basicstyle=\tt]{arborescence.txt}

\section{Déploiement d'applications}

Le déploiement d'applications sur les différentes plateformes nécessite quelques changements mineurs 
de l'application en fonction de la cible visée.

Tout d'abord, Enyo fournit un ``bootplate'' décrivant l'organisation standard d'une application.
Ce dossier contient un script ``deploy'' permettant de préparer l'application au 
déploiement ainsi que l'analyse du programme et la suppression des fichiers inutilisés.

A partir de là, pour un déploiement sur tablette HP, par exemple, il est nécessaire d'obtenir les outils
développeur de la tablette permettant de ``packagé'' l'application.
Pour un déploiement sur téléphone (``android'', ``iOs'', etc.), il est nécessaire de télécharger
``PhoneGap'' \href{http://www.phonegap.com/developer}{[lien]} qui propose ce genre de ``packaging''.
Il est toutefois nécessaire de fournir un noeud \emph{meta} dans le document Html principal de l'application
selon la plateforme.
Les détails sont accessibles à cette adresse : \\
\url{https://github.com/enyojs/enyo/wiki/Platform-Specific-Deployment}

\section{Exemples de programme}

Les exemples d'applications se situent dans le dossier ``exemples'' à la racine du projet.

On pourra y trouver quelques applications JavaScript ainsi que des jeux utilisant la bibliothèque inspiré des exemples contenu sur le cédérom 
du manuel ``DAOC''\cite{DAOC}.

\tableofcontents
\nocite{*}
\bibliographystyle{plain}
\bibliography{biblio}

\end{document}
